A configuração de diversas opções e principalmente dos elementos pré-textuais é realizada com comandos específicos inseridos antes do comando \comando{begin\{document\}}. As informações do documento são configuradas através dos comandos:
\begin{description}

 \item[\comando{titulo\{T\}}] Título do trabalho (substitua T pelo título do trabalho);

 \item[\comando{autor[REF]\{N\}}] Nome do autor do trabalho (onde REF é como o nome do autor é referenciado e N é o nome do autor);

 \item[\comando{orientador\{T\}\{O\}}] Nome do professor orientador do trabalho. Caso seja uma orientadora pode ser usado o comando \comando{orientador[Orientadora:]\{T\}\{O\}} (sendo que T é a titulação do professor e O é o nome do orientador);

 \item[\comando{coorientador\{T\}\{C\}}] Nome do professor coorientador do trabalho. Caso seja uma coorientadora pode ser usado um comando análogo a definição de orientadora  empregando o comando \comando{coorientador[Coorientadora:]\{T\}\{C\}}(sendo que T é a titulação do professor e C é o nome do orientador);

 
 \item[\comando{curso\{EP\}\{NP\}}] Dados do programa de Pós-Graduação (onde EP é a especialidade que será atribuída ao pós-graduando e NP é o nome do programa de pós-graduação.  Exemplo: \comando{curso\{Ciências -- Ciências de Computação e Matemática Computacional\}\{Ciências de Computação e Matemática Computacional\}};
 
 \item[\comando{data\{dia\}\{mês\}\{ano\}}] Configuração da data do depósito do documento;

 \item[\comando{textoresumo\{TR\}\{PC\}}] Texto do resumo (TR) e palavras-chave (PC) do documento sendo separadas por vírgula. Se o idioma do resumo for diferente do declarado no documento, pode ser usado o comando \comando{textoresumo[L]\{TR\}\{PC\}} (sendo que L é a linguagem do resumo).
 
\end{description}