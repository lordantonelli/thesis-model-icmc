
Corpos flutuantes são elementos não textuais como figuras e tabelas que complementam as informações do texto. Neste capítulo são expostos breves exemplos dos corpos flutuantes disponíveis na classe \textit{icmc}.

Na Seção \ref{secao:figuras} é mostrado como inserir figuras, a Seção \ref{secao:tabelas_e_quadros} explica como incluir tabelas e quadros e a Seção \ref{secao:algoritmos_e_codigos} demostra como trabalhar com algoritmos e códigos fontes.

\section{Figuras}
\label{secao:figuras}

A inserção de figuras é realizada normalmente através do comando \comando{begin\{figure\}}. Na Figura \ref{figura:exemplo_grafo} é mostrado um exemplo de grafo com o pacote \textit{xy}. Já a Figura \ref{figura:logomarca_usp} exibe a logomarca da USP com o pacote \textit{graphicx}. De acordo com as normas ABNT a lista de figuras é um elemento opcional do documento, para incluí-la é preciso inserir o comando \comando{incluidelistafiguras} antes do início do documento.

Desde 2012, deve ser incorporado ao corpo flutuante do tipo figura, além da legenda, a fonte de onde esta foi extraída. Se a figura foi confeccionada pelo próprio autor, deve se colocar " o autor".

\begin{figure}[htb]
\caption{Exemplo de grafo}
\label{figura:exemplo_grafo}
\centering
\begin{scriptsize}
$$
\xymatrix@R20pt@C10pt{
 & & & & vr \ar[dlll] \ar[dl] \ar[d] \ar[dr] \ar[drr] \ar[drrr] & & & \\
 & (a_3, b_2, c_1) \ar[d]^{\varphi_2} \ar[dl]_{\varphi_1} & & (a_3, b_2, c_2) \ar[d]^{\varphi_2} \ar[dl]_{\varphi_1} & (a_1, b_1, c_1) & (a_1, b_1, c_2) & (a_1, b_2, c_1) & (a_1, b_2, c_2) \\
 (a_2, b_2, c_1) \ar[dr]_{\varphi_3} & (a_3, b_1, c_1) \ar[d]^{\varphi_1} & (a_2, b_2, c_2) \ar[dr]_{\varphi_3} & (a_3, b_1, c_2) \ar[d]^{\varphi_1} & & & & \\
& (a_2, b_1, c_1)  & & (a_2, b_1, c_2) & & & & \\
}
$$
\end{scriptsize}
\legend{Fonte: o  autor.}
\end{figure}

\begin{figure}[htb]
 \caption{Logomarca da USP}
 \label{figura:logomarca_usp}
 \centering
 \includegraphics[scale=0.3]{images/usp-logo}
 \legend{Fonte: Universidade de São Paulo \cite{usp:logo}}
\end{figure}


\section{Tabelas e Quadros}
\label{secao:tabelas_e_quadros}

A inserção de tabelas e quadros é feita de forma semelhante a inserção de figuras, porém são utilizados os ambientes \textit{table} e \textit{quadro}. A principal diferença entre tabelas e quadros, de acordo com \citeonline{silveira:2006:manual_tcc}, é que as tabelas são destinadas para informações numéricas e os quadros são mais adequados para informações textuais.

 Como exemplos foram inseridas a Tabela \ref{tabela:lista_produtos} que exibe uma de lista de produtos e a Tabela \ref{tabela:populacao_america_sul} que mostra a população dos países da América do Sul. Foi inserido também o Quadro \ref{quadro:editores_texto_livres} com alguns editores que podem ser usados para se trebalhar com Latex para demonstrar a inserção de quadros.

 A lista de tabelas também é um elemento opcional que pode ser incluída com o comando \comando{incluidelistatabelas} antes do início do documento. O mesmo acontece com a lista de quadros que pode ser incluída com o comando \comando{incluidelistaquadros}.

\begin{table}[htb]
\centering
\caption{Lista de produtos}
\label{tabela:lista_produtos}
\begin{tabularx}{\textwidth}{X|l|r|r|r} \hline
Produto      & Unidade & Preço (R\$) & Quantidade & Total (R\$) \\ \hline
Arroz        & Kg      & 2,00        & 550        & 1.100,00    \\
Óleo de Soja & L       & 2,50        & 500        & 750,00      \\
Açucar       & Kg      & 3,00        & 100        & 300,00      \\ \hline
\end{tabularx}
\end{table}

\begin{table}[htb]
\centering
\caption{População dos países da América do Sul} \label{tabela:populacao_america_sul}
\begin{tabular}{r|l|r}        \hline
Código  & País            & População   \\ \hline
1       & Brasil          & 191.480.630 \\
2       & Argentina       &  39.934.100 \\
3       & Colômbia        &  46.741.100 \\
4       & Paraguai        &   9.694.200 \\
5       & Uruguai         &   3.350.500 \\
6       & Peru            &  28.221.500 \\
7       & Equador         &  13.481.200 \\
8       & Bolívia         &   9.694.200 \\
9       & Venezuela       &  28.121.700 \\
10      & Chile           &  16.803.000 \\ \hline
\end{tabular}

\legend{Fonte: \citeonline{wikipedia:2011:america_sul}.}
\end{table}

\begin{quadro}[htb]
\caption{Editores de Texto Livres}
\label{quadro:editores_texto_livres}
\centering
\begin{tabular}{|l|l|r|}        \hline
Editor     & Multiplataforma & Específico para Latex \\ \hline
Kwriter    & Sim             & Não                   \\
Texmaker   & Sim             & Sim                   \\
Kile       & Sim             & Sim                   \\
Geany      & Sim             & Não                   \\ \hline
\end{tabular}
\end{quadro}

\section{Algoritmos e Códigos}
\label{secao:algoritmos_e_codigos}

Além dos corpos flutuantes convencionais para inserir figuras (\comando{begin\{figure\}}) e tabelas (\comando{begin\{figure\}}), a classe \textit{icmc} possui mais dois tipos de corpos flutuantes um para algoritmos (\comando{begin\{algoritmo\}}) e outro para códigos (\comando{begin\{codigo\}}). A utilização de um ou de outro fica a critério do usuário. Como exemplo temos o Algoritmo \ref{algoritmo:mdc1} que calcula o máximo divisor comum entre dois números e os Códigos \ref{codigo:notas_alunos} e \ref{codigo:metodo_leitura} que são uma consulta na \textit{Structured Query Language (SQL)}\sigla{SQL}{Structured Query Language} e uma sobrotina em \textit{Java}.

%\begin{algoritmo}[htb]
\begin{algoritmo}
%\begin{algorithmic}[1]
\caption{Algoritmo para cálculo de máximo divisor comum MDC($n_1$,$n_2$)}
\label{algoritmo:mdc1}

 \KwIn{Dois números inteiros ($n_1, n_2$)}
 \If(\tcp*[f]{Garante que o maior número seja $n_1$}){$n_2 > n_1$}
   {troca valores de $n_1$ e $n_2$}
 \Repeat{$r > 0$}{
    $r \leftarrow$ resto da divisão de $n_1$ por $n_2$
    $n_1 \leftarrow n_2$
    $n_2 \leftarrow r$
 }
 \Return $n_1$
%\end{algorithmic}
\end{algoritmo}
%\end{algoritmo}

%\begin{codigo}[htb]
%\caption{Consulta SQL}
%\label{codigo:notas_alunos}
%\hrule
\begin{codigo}[caption = {Consulta SQL}, label={codigo:notas_alunos},language=SQL, breaklines=true]
SELECT a.nome_aluno AS aluno,
       d.nome_disciplina AS disciplina,
       m.nota AS nota
FROM aluno AS a,
     disciplina AS d,
     matriculado AS m
WHERE a.id_aluno = m.id_aluno
  AND d.id_disciplina = m.id_disciplina
ORDER BY a.nome_aluno, d.nome_disciplina;
\end{codigo}
%\end{codigo}

%\begin{codigo}[htb]
%\caption{Subrotina para obter uma entrada do usuário}
%\label{codigo:metodo_leitura}
%\hrule
\begin{codigo}[caption={Subrotina para obter uma entrada do usuário}, label={codigo:metodo_leitura}, language=Java, breaklines=true]
public static String Leitura(){
    BufferedReader reader = new BufferedReader(new InputStreamReader(System.in));
    try {
        return reader.readLine(); // Lê uma linha pelo teclado
    } catch (IOException e) {
        e.printStackTrace();
        return "";
    }
}
\end{codigo}
%\end{codigo}

Existem diversos outros pacotes disponíveis para escrever algoritmos e códigos. Nos exemplos anteriormente foram utilizados o pacote \textit{algpseudocode} e \textit{listings}. O pacote \textit{algpseudocode} é usado para escrever algoritmos em alto nível \cite{janos:2005:algpseudocode}. Já o pacote \textit{listings} serve para escrever os códigos em diversas linguagens de programação \cite{moses:2006:listings}.

Caso sejam utilizados os ambientes de algoritmos e código podem ser incluídos os comandos \comando{incluidelistaalgoritmos} e \comando{incluidelistacodigos} antes do \comando{begin\{document\}} para que a lista de algoritmos e a lista de código sejam criadas.


\section{Ambientes Matemáticos}

A classe \textit{icmc} provê os seguintes ambientes matemáticos:
\begin{itemize}
 \item Teoremas (\comando{begin\{teorema\}[\ ]} ... \comando{begin\{teorema\}});
 \item Proposição (\comando{begin\{proposicao\}[\ ]} ... \comando{begin\{proposicao\}});
 \item Lema (\comando{begin\{lema\}[\ ]} ... \comando{begin\{lema\}});
 \item Corolário (\comando{begin\{corolario\}[\ ]} ... \comando{begin\{corolario\}});
 \item Exemplo (\comando{begin\{exemplo\}[\ ]} ... \comando{begin\{exemplo\}});
 \item Observação (\comando{begin\{observacao\}[\ ]} ... \comando{begin\{observacao\}});
 \item Definição (\comando{begin\{definicao\}[\ ]} ... \comando{begin\{definicao\}});
 \item demonstracao (\comando{begin\{demonstracao\}[\ ]} ... \comando{begin\{demonstracao\}}).
\end{itemize}

Abaixo temos um exemplo de proposição com sua demonstração:
\begin{proposicao}
 Sejam $a$ e $b$ reais, tais que $0<a<b$. Então $a^2<b^2$.
\end{proposicao}
\begin{demonstracao}
 Pela hipótese concluímos que $(b+a)>0$ e $(b-a)>0$.

Como $b^2-a^2=(b+a)(b-a)$ concluímos que $b^2-a^2>0$, ou seja, $a^2<b^2$.
\end{demonstracao}

Neste documento tratamos brevemente apenas dos ambientes mencionados anteriormente. Contudo, para escrever expressões matemáticas complexas é preciso estudar uma documentação mais específica como em \citeonline{cassagojr:1997:amslatex}.

Alguns dos ambientes matemáticos da classe \textit{icmc} podem ser usados também para outras finalidades como exemplos e definições.