% ------------------------------------------------------------------------
% ------------------------------------------------------------------------
% ICMC: Modelo de Trabalho Acadêmico (tese de doutorado, dissertação de
% mestrado e trabalhos monográficos em geral) em conformidade com 
% ABNT NBR 14724:2011: Informação e documentação - Trabalhos acadêmicos -
% Apresentação
% ------------------------------------------------------------------------
% ------------------------------------------------------------------------

% Opções: 
%   Qualificação         = qualificacao 
%   Curso                = doutorado/mestrado
%   Situação do trabalho = pre-defesa/pos-defesa (exceto para qualificação)
% -- opções do pacote babel --
% Idioma padrão = brazil
	%french,	    	% idioma adicional para hifenização
	%spanish,			% idioma adicional para hifenização
	%english,			% idioma adicional para hifenização
	%brazil				% o último idioma é o principal do documento
\documentclass[mestrado, spanish, english, brazil]{packages/icmc}

% ---
% Pacotes Opcionais
% ---
\usepackage{mathptmx}           % Usando times para equações, opcional
\usepackage{rotating}           % Usado para rotacionar o texto
\usepackage{algpseudocode}      % Pacote para escrever algoritmos em pseudo-código
\usepackage[all,knot,arc,import,poly]{xy}   % Pacote para desenhos gráficos
% Este pacote pode conflitar com outros pacotes gráficos como o ``pictex''
% Então é necessário usar apenas um dos pacotes conflitantes


% ---
% Informações de dados para CAPA e FOLHA DE ROSTO
% ---
\titulo{Modelo de Teses e Dissertações em \LaTeX~ do ICMC}
\autor[Antonelli, H. L.]{Humberto Lidio Antonelli}
\local{São Carlos -- SP}
\data{01}{12}{2014} % Data do depósito
\orientador[Orientadora:]{Profa. Dra.}{Renata Pontin de Mattos Fortes}
%\coorientador{Prof. Dr.}{Fulano de tal}
\instituicao{Instituto de Ciências Matemáticas e de Computação (ICMC/USP)}
\curso{Ciências de Computação e Matemática Computacional}
% ---


% ---
% RESUMOS
% ---

% Resumo em português
% conter no máximo 500 palavras
\textoresumo{
    Este trabalho é um breve modelo  para a escrita de monografias de qualificação, dissertações e teses utilizando o ambiente \LaTeX, de acordo com as normas exigidas pelo Instituto de Ciências Matemáticas e de Computação (ICMC), da Universidade de São Paulo (USP). Para a confecção deste modelo foi utilizado a última versão (1.9.2) do pacote de classes \textit{abnTeX2} que segue as normas da Associação Brasileira de Normas Técnicas. A elaboração de uma monografia, dissertação ou tese pode ser feita sobrescrevendo o conteúdo deste modelo. 
    }{Modelo, Monografia de qualificação, Dissertação, Tese, Latex}

% ---
% resumo em inglês
% ---
\textoresumo[english]{
    This paper is a brief model for writing qualification monographs, dissertations and thesis using \LaTeX environment, in accordance with the standards required by the Institute of Mathematics and Computer Sciences (ICMC), University of São Paulo (USP). For making this model, the latest version (1.9.2) \textit{abnTeX2} classes package was used. This package follow the rules of the Brazilian Association of Technical Standards. A drafting a monograph, dissertation or thesis can be done by overwriting the contents of this model.
    }{Template, Qualification monograph, Dissertation, Thesis, Latex}

% ---
% resumo em espanhol
% ---
\textoresumo[spanish]{
    Este trabajo es un breve modelo para la redacción de monografías de cualificación, disertaciones y tesis utilizando el ambiente LaTeX, de acuerdo a las normas exigidas por el Instituto de Ciencias Matemáticas y Computación (ICMC) de la Universidad de São Paulo (USP). Para la confección de este modelo fue utilizada la última versión (1.9.1) del paquete de clases abnTeX2 que sigue las normas de la Asociación Brasilera de Normas Técnicas. La elaboración de una monografía, disertación o tesis puede ser realizada reescribiendo el contenido de este modelo.
    }{Template, Monografías de cualificación, Disertacion, Tesis, Latex}
    
    
% ---
% Configurações de aparência do PDF final
% ---
% alterando o aspecto da cor azul
\definecolor{blue}{RGB}{41,5,195}

% informações do PDF
\makeatletter
\hypersetup{
     	pagebackref=true,
		pdftitle={\@title}, 
		pdfauthor={\@author},
    	pdfsubject={\imprimirpreambulo},
	    pdfcreator={LaTeX with abnTeX2/ICMC-USP},
		pdfkeywords={\palavraschave}, 
		colorlinks=true,       		% false: boxed links; true: colored links
    	linkcolor=blue,          	% color of internal links
    	citecolor=blue,        		% color of links to bibliography
    	filecolor=magenta,      	% color of file links
		urlcolor=blue,
		bookmarksdepth=4
}
\makeatother
% --- 

% ----------------------------------------------------------
% ELEMENTOS PRÉ-TEXTUAIS
% ----------------------------------------------------------

% Inserir a ficha catalográfica
%\incluifichacatalografica[tex/fichaCatalografica.pdf]
%\incluifichacatalografica


% Inserir folha de aprovação
%\input{tex/folha-aprovacao}

% DEDICATÓRIA / AGRADECIMENTO / EPÍGRAFE
\textodedicatoria*{tex/pre-textual/dedicatoria}
\textoagradecimentos*{tex/pre-textual/agradecimentos}
\textoepigrafe*{tex/pre-textual/epigrafe}

% Inclui a lista de figuras
\incluilistadefiguras

% Inclui a lista de tabelas
\incluilistadetabelas

% Inclui a lista de quadros
\incluilistadequadros

% Inclui a lista de algoritmos
\incluilistadealgoritmos

% Inclui a lista de códigos
\incluilistadecodigos

% Inclui a lista de siglas e abreviaturas
\incluilistadesiglas

% Inclui a lista de símbolos
\incluilistadesimbolos

% ----
% Início do documento
% ----
\begin{document}

% ----------------------------------------------------------
% ELEMENTOS TEXTUAIS
% ----------------------------------------------------------
\textual

\chapter{Introdução}
\label{chapter:introducao}
% Comando simples para exibir comandos Latex no texto
\newcommand{\comando}[1]{\textbf{$\backslash$#1}}

Este documento explica brevemente como trabalhar com a classe \LaTeX~\textit{icmc} para confeccionar trabalhos acadêmicos seguindo as normas da \sigla{ABNT}{Associação Brasileira de Normas Técnicas} e as \aspas{\textit{Diretrizes para apresentação de dissertações e teses da USP: documento eletrônico e impresso. Parte I (ABNT)}}, publicado pelo \sigla{SIBi}{Sistema Integrado de Bibliotecas} USP. O presente manual também atende as exigências prevista no regimento do Programa de Pós-graduação em \sigla{CCMC}{Ciências da Computação e Matemática Computacional} do \sigla{ICMC}{Instituto de Ciências Matemáticas e de Computação} da \sigla{USP}{Universidade de São Paulo}.


A classe \textit{icmc} foi construída com base na última versão da classe \textit{abntex2} e do pacote \textit{abntex2cite}. Portanto, este documento exemplifica a elaboração de trabalho
acadêmico (tese, dissertação e outros do gênero) produzido conforme a ABNT NBR
14724:2011 \textit{Informação e documentação - Trabalhos acadêmicos - Apresentação}.

Assim, é altamente recomendável que seja consultada a documentação do \textit{abntex2}\footnote{http://abntex.net.br}. A classe \textit{abntex2} foi desenvolvida para facilitar a escrita de documentos seguindo as normas da ABNT no ambiente \LaTeX\;\cite{frasson:2005:classe_abnt}.

Todo o trabalho de pesquisa e ajustes da presente classe \LaTeX~\emph{icmc} foram feitos pelo aluno mestrado do Programa de Pós-graduação em Ciência da Computação e Matemática Computacional, Humberto Lidio Antonelli, durante a confecção da sua monografia de qualificação.

O requisito básico para utilização da classe \textit{icmc} é criar um documento desta classe com o comando
\comando{documentclass[@parameters]\{icmc\}} e ter, no diretório de trabalho, o arquivo \emph{icmc.cls} presente. Entretanto, recomenda-se fortemente manter a estrutura de diretório inicial fornecida por este modelo. Além disso, para que o documento esteja em conformidade com as normas exigidas pelo programa de Pós-Graduação, o \textbf{projeto deve ser compilado utilizando \textit{XeLaTeX} ou \textit{LuaLaTeX}}. Esse processo de compilação é necessário para que as fontes externas utilizadas para gerar a capa sejam incluídas.

Os parâmetros possíveis utilizados pelo \comando{documentclass} são:
\begin{description}
\item[qualificacao] Exclusivamente para monografias de qualificação em geral;
\item[mestrado / doutorado] Identifica o curso ao qual o aluno pertence, sendo utilizado apenas uma das duas opcões disponíveis. O valor padrão é \textbf{doutorado};
\item[pre-defesa / pos-defesa] Identifica a situação do documento (exceto para qualificação), sedo necessário apenas uma das duas opções. O valor padrão é \textbf{pos-defesa};
\item[impressao] Gera exclusivamente uma versão para impressão do documento;
\item[french, spanish, english, brazil] Adiciona o idioma para correta hifenização correta no documento. Os idiomas bases para o modelo (português e inglês) não precisam ser declarados.
\end{description}


\chapter{Instalando o abnTeX2}
\label{chapter:instalando-abntex}

A instalação do \emph{abnTeX2} varia de acordo com o sistema operacional empregado pelo usuário. Aqui serão apresentadas as formas de instalação nos sistemas mais utilizados atualmente, a saber: Linux (Ubuntu 12.04), Mac OS X e Windows 7

\section{Linux (Ubuntu 12.04)}

Se você já instalou o Tex Live via apt-get, basta seguir os seguintes comandos:

\begin{enumerate}

\item Baixe os arquivos de instalação do abnTeX2 (\url{http://code.google.com/p/abntex2/downloads/list}). Nesse link você também encontra a documentação e exemplos de uso.
\item Extraia o conteúdo do arquivo baixado na pasta texmf local, geralmente /usr/local/share/texmf. 
\item Em um Terminal: extraia o ZIP: \emph{unzip abntex2.tds.zip} em qualquer local;
\item copie o conteúdo extraído para o destino: \emph{cp abntex2/* /usr/local/share/texmf};
\item Em um Terminal digite: \emph{sudo texhash}
\item Pronto!
\end{enumerate}

\section{Mac OS}

Primeiramente, deve-se abrir o terminal do Mac que pode ser encontrado em Aplicativos/Utilitários - buscando pelo Finder.  E seguir os comandos abaixo:
\begin{enumerate}
\item Baixe os arquivos de instalação do abnTeX2 (\url{http://code.google.com/p/abntex2/downloads/list}). Nesse link você também encontra a documentação e exemplos de uso.
\item Extraia o conteúdo do arquivo baixado na pasta \emph{texmf} local, geralmente \emph{/usr/local/texlive/texmf-local}
\item Em um Terminal digite: \emph{sudo texhash}
\item Pronto!
\end{enumerate}
 
 \section{Windows 7}

\subsection{Instalar/atualizar pelo Package Manager (recomendado)}

Geralmente o abnTeX2 é baixado e instalado automaticamente pelo MiKTeX quando o usuário compila pela primeira vez um dos modelos do abnTeX2. Porém, caso isso não ocorra, siga os passos seguintes:

\begin{enumerate}
\item Clique em Iniciar/Start -> Todos os Programas/All Programs -> MiKTeX -> Package Manager;
\item Clique em Repository / Synchronize;
\item Clique com o botão direito sobre \emph{abntex2} na lista e selecione Install (ou Update, caso já esteja instalado);
\item Pronto!
\end{enumerate}

\subsection{Instalar/atualizar manualmente}

Você apenas precisará utilizar a instalação manual no caso de:

\begin{enumerate}
\item o abnTeX2 não estar na lista de pacotes do MiKTeX por alguma razão;
\item você não poder utilizar uma conexão com a Internet no momento da instalação;
\item a versão do abnTeX2 no MiKTeX estar desatualizada em relação à versão disponível no CTAN.
\end{enumerate}
Em qualquer caso, lembre-se de remover uma eventual instalação anterior do abnTeX2 . Se houver instalado pelo Package Manager, remova o abnTeX2 também por ele.

Passos para instalação manual do abnTeX2 no MiKTeX:

\begin{enumerate}
\item Baixe os arquivos de instalação do abnTeX2 (abntex2.tds-vX.X.zip). Nesse link você também encontra a documentação e exemplos de uso.
\item Extraia o conteúdo do arquivo baixado em uma pasta qualquer;
\item Você pode criar uma pasta abntex2, por exemplo, em $C:\backslash abntex2\backslash$;
\item Consulte http://www.tex.ac.uk/cgi-bin/texfaq2html?label=install-where para outras informações;
\item Clique em Iniciar/Start -> Todos os Programas/All Programs -> MiKTeX -> Settings;
\item Na aba Roots, adicione o diretório recém criado;
\item Na aba General, clique em Refresh FNDB, OU, se preferir, em um Terminal digite initexmf --update-fndb;
\item Pronto!
\end{enumerate}

\chapter{Configuração dos Elementos Pré-Textuais}
\label{chapter:config-pre-textual}
A configuração de diversas opções e principalmente dos elementos pré-textuais é realizada com comandos específicos inseridos antes do comando \comando{begin\{document\}}. As informações do documento são configuradas através dos comandos:
\begin{description}
 \item[\comando{titulo\{T\}}] Título do trabalho (substitua T pelo título do trabalho);
 \item[\comando{autor\{N\}}] Nome do autor do trabalho (onde N é o nome do autor);
 \item[\comando{orientador\{O\}}] Nome do professor orientador do trabalho. Caso seja uma orientadora pode ser usado o comando \comando{orientador[Orientadora:$\backslash\backslash$]\{O\}} (sendo que O é o nome do orientador ou orientadora);
 \item[\comando{coorientador\{C\}}] Nome do professor coorientador do trabalho. Caso seja uma coorientadora pode ser usado um comando análogo a definição de orientadora  empregando o comando \comando{coorientador[Coorientadora:$\backslash\backslash$]\{C\}}(sendo que C é o nome do orientador ou orientadora);
 \item[\comando{departamento\{D\}}] Nome do departamento sob o qual está o curso do aluno (substituindo D pelo nome do departamento);
 
 \item[\comando{curso\{MC\}\{NC\}\{GC\}}] Dados do curso, modalidade do curso(MC), nome do curso(NC) e grau obtido com o curso(GC). Exemplo: \comando{curso\{Bacharelado\}\{Ciência da Computação\}\{Bacharel\}};
 
 \item[Membros da banca avaliadora] Os membros da banca avaliadora constarão na folha de aprovação e são definidos através dos comandos \comando{orientador\{\}}, \comando{coorientador\{\}} (caso exista) e \comando{membrobanca\{\}}. O orientador será o primeiro membro da folha de aprovação, o coorientador será o segundo (se existir), seguidos pelos membros definidos pelos comandos\comando{membrobanca\{\}} quantas vezes forem necessárias para se completar a banca examinadora, que deve possui pelo menos dois membros, além do orientador e coorientador. A definição destes últimos seguem o mesmo formato:
 
\comando{membrobanca\{NM\}\{IM\}} (onde NM é o nome do membro e IM é a instituição do membro);
 \item[\comando{data\{dia\}\{mês (por extenso)\}\{ano\}}] Configuração da data do documento que aparecerá na folha de aprovação;
 \item[\comando{textoresumo\{TR\}\{PC\}}] Texto do resumo (TR) e palavras chaves (PC) do documento. Cada palavra chave deve ser inserida com o comando \comando{palavrachave\{P\}}, onde P é a palavra chave.
\end{description}

\chapter{Corpos Flutuantes}
\label{chapter:corpos-flutuantes}

Corpos flutuantes são elementos não textuais como figuras e tabelas que complementam as informações do texto. Neste capítulo são expostos breves exemplos dos corpos flutuantes disponíveis na classe \textit{icmc}.

Na Seção \ref{secao:figuras} é mostrado como inserir figuras, a Seção \ref{secao:tabelas_e_quadros} explica como incluir tabelas e quadros e a Seção \ref{secao:algoritmos_e_codigos} demostra como trabalhar com algoritmos e códigos fontes.

\section{Figuras}
\label{secao:figuras}

A inserção de figuras é realizada normalmente através do comando \comando{begin\{figure\}}. Na Figura \ref{figura:exemplo_grafo} é mostrado um exemplo de grafo com o pacote \textit{xy}. Já a Figura \ref{figura:logomarca_usp} exibe a logomarca da USP com o pacote \textit{graphicx}. De acordo com as normas ABNT a lista de figuras é um elemento opcional do documento, para incluí-la é preciso inserir o comando \comando{incluidelistafiguras} antes do início do documento.

Desde 2012, deve ser incorporado ao corpo flutuante do tipo figura, além da legenda, a fonte de onde esta foi extraída. Se a figura foi confeccionada pelo próprio autor, deve se colocar " o autor".

\begin{figure}[htb]
\caption{Exemplo de grafo}
\label{figura:exemplo_grafo}
\centering
\begin{scriptsize}
$$
\xymatrix@R20pt@C10pt{
 & & & & vr \ar[dlll] \ar[dl] \ar[d] \ar[dr] \ar[drr] \ar[drrr] & & & \\
 & (a_3, b_2, c_1) \ar[d]^{\varphi_2} \ar[dl]_{\varphi_1} & & (a_3, b_2, c_2) \ar[d]^{\varphi_2} \ar[dl]_{\varphi_1} & (a_1, b_1, c_1) & (a_1, b_1, c_2) & (a_1, b_2, c_1) & (a_1, b_2, c_2) \\
 (a_2, b_2, c_1) \ar[dr]_{\varphi_3} & (a_3, b_1, c_1) \ar[d]^{\varphi_1} & (a_2, b_2, c_2) \ar[dr]_{\varphi_3} & (a_3, b_1, c_2) \ar[d]^{\varphi_1} & & & & \\
& (a_2, b_1, c_1)  & & (a_2, b_1, c_2) & & & & \\
}
$$
\end{scriptsize}
\legend{Fonte: o  autor.}
\end{figure}

\begin{figure}[htb]
 \caption{Logomarca da USP}
 \label{figura:logomarca_usp}
 \centering
 \includegraphics[scale=0.3]{images/usp-logo}
 \legend{Fonte: Universidade de São Paulo \cite{usp:logo}}
\end{figure}


\section{Tabelas e Quadros}
\label{secao:tabelas_e_quadros}

A inserção de tabelas e quadros é feita de forma semelhante a inserção de figuras, porém são utilizados os ambientes \textit{table} e \textit{quadro}. A principal diferença entre tabelas e quadros, de acordo com \citeonline{silveira:2006:manual_tcc}, é que as tabelas são destinadas para informações numéricas e os quadros são mais adequados para informações textuais.

 Como exemplos foram inseridas a Tabela \ref{tabela:lista_produtos} que exibe uma de lista de produtos e a Tabela \ref{tabela:populacao_america_sul} que mostra a população dos países da América do Sul. Foi inserido também o Quadro \ref{quadro:editores_texto_livres} com alguns editores que podem ser usados para se trebalhar com Latex para demonstrar a inserção de quadros.

 A lista de tabelas também é um elemento opcional que pode ser incluída com o comando \comando{incluidelistatabelas} antes do início do documento. O mesmo acontece com a lista de quadros que pode ser incluída com o comando \comando{incluidelistaquadros}.

\begin{table}[htb]
\centering
\caption{Lista de produtos}
\label{tabela:lista_produtos}
\begin{tabularx}{\textwidth}{X|l|r|r|r} \hline
Produto      & Unidade & Preço (R\$) & Quantidade & Total (R\$) \\ \hline
Arroz        & Kg      & 2,00        & 550        & 1.100,00    \\
Óleo de Soja & L       & 2,50        & 500        & 750,00      \\
Açucar       & Kg      & 3,00        & 100        & 300,00      \\ \hline
\end{tabularx}
\end{table}

\begin{table}[htb]
\centering
\caption{População dos países da América do Sul} \label{tabela:populacao_america_sul}
\begin{tabular}{r|l|r}        \hline
Código  & País            & População   \\ \hline
1       & Brasil          & 191.480.630 \\
2       & Argentina       &  39.934.100 \\
3       & Colômbia        &  46.741.100 \\
4       & Paraguai        &   9.694.200 \\
5       & Uruguai         &   3.350.500 \\
6       & Peru            &  28.221.500 \\
7       & Equador         &  13.481.200 \\
8       & Bolívia         &   9.694.200 \\
9       & Venezuela       &  28.121.700 \\
10      & Chile           &  16.803.000 \\ \hline
\end{tabular}

\legend{Fonte: \citeonline{wikipedia:2011:america_sul}.}
\end{table}

\begin{quadro}[htb]
\caption{Editores de Texto Livres}
\label{quadro:editores_texto_livres}
\centering
\begin{tabular}{|l|l|r|}        \hline
Editor     & Multiplataforma & Específico para Latex \\ \hline
Kwriter    & Sim             & Não                   \\
Texmaker   & Sim             & Sim                   \\
Kile       & Sim             & Sim                   \\
Geany      & Sim             & Não                   \\ \hline
\end{tabular}
\end{quadro}

\section{Algoritmos e Códigos}
\label{secao:algoritmos_e_codigos}

Além dos corpos flutuantes convencionais para inserir figuras (\comando{begin\{figure\}}) e tabelas (\comando{begin\{figure\}}), a classe \textit{icmc} possui mais dois tipos de corpos flutuantes um para algoritmos (\comando{begin\{algoritmo\}}) e outro para códigos (\comando{begin\{codigo\}}). A utilização de um ou de outro fica a critério do usuário. Como exemplo temos o Algoritmo \ref{algoritmo:mdc1} que calcula o máximo divisor comum entre dois números e os Códigos \ref{codigo:notas_alunos} e \ref{codigo:metodo_leitura} que são uma consulta na \textit{Structured Query Language (SQL)}\sigla{SQL}{Structured Query Language} e uma sobrotina em \textit{Java}.

%\begin{algoritmo}[htb]
\begin{algoritmo}
%\begin{algorithmic}[1]
\caption{Algoritmo para cálculo de máximo divisor comum MDC($n_1$,$n_2$)}
\label{algoritmo:mdc1}

 \KwIn{Dois números inteiros ($n_1, n_2$)}
 \If(\tcp*[f]{Garante que o maior número seja $n_1$}){$n_2 > n_1$}
   {troca valores de $n_1$ e $n_2$}
 \Repeat{$r > 0$}{
    $r \leftarrow$ resto da divisão de $n_1$ por $n_2$
    $n_1 \leftarrow n_2$
    $n_2 \leftarrow r$
 }
 \Return $n_1$
%\end{algorithmic}
\end{algoritmo}
%\end{algoritmo}

%\begin{codigo}[htb]
%\caption{Consulta SQL}
%\label{codigo:notas_alunos}
%\hrule
\begin{codigo}[caption = {Consulta SQL}, label={codigo:notas_alunos},language=SQL, breaklines=true]
SELECT a.nome_aluno AS aluno,
       d.nome_disciplina AS disciplina,
       m.nota AS nota
FROM aluno AS a,
     disciplina AS d,
     matriculado AS m
WHERE a.id_aluno = m.id_aluno
  AND d.id_disciplina = m.id_disciplina
ORDER BY a.nome_aluno, d.nome_disciplina;
\end{codigo}
%\end{codigo}

%\begin{codigo}[htb]
%\caption{Subrotina para obter uma entrada do usuário}
%\label{codigo:metodo_leitura}
%\hrule
\begin{codigo}[caption={Subrotina para obter uma entrada do usuário}, label={codigo:metodo_leitura}, language=Java, breaklines=true]
public static String Leitura(){
    BufferedReader reader = new BufferedReader(new InputStreamReader(System.in));
    try {
        return reader.readLine(); // Lê uma linha pelo teclado
    } catch (IOException e) {
        e.printStackTrace();
        return "";
    }
}
\end{codigo}
%\end{codigo}

Existem diversos outros pacotes disponíveis para escrever algoritmos e códigos. Nos exemplos anteriormente foram utilizados o pacote \textit{algpseudocode} e \textit{listings}. O pacote \textit{algpseudocode} é usado para escrever algoritmos em alto nível \cite{janos:2005:algpseudocode}. Já o pacote \textit{listings} serve para escrever os códigos em diversas linguagens de programação \cite{moses:2006:listings}.

Caso sejam utilizados os ambientes de algoritmos e código podem ser incluídos os comandos \comando{incluidelistaalgoritmos} e \comando{incluidelistacodigos} antes do \comando{begin\{document\}} para que a lista de algoritmos e a lista de código sejam criadas.


\section{Ambientes Matemáticos}

A classe \textit{icmc} provê os seguintes ambientes matemáticos:
\begin{itemize}
 \item Teoremas (\comando{begin\{teorema\}[\ ]} ... \comando{begin\{teorema\}});
 \item Proposição (\comando{begin\{proposicao\}[\ ]} ... \comando{begin\{proposicao\}});
 \item Lema (\comando{begin\{lema\}[\ ]} ... \comando{begin\{lema\}});
 \item Corolário (\comando{begin\{corolario\}[\ ]} ... \comando{begin\{corolario\}});
 \item Exemplo (\comando{begin\{exemplo\}[\ ]} ... \comando{begin\{exemplo\}});
 \item Observação (\comando{begin\{observacao\}[\ ]} ... \comando{begin\{observacao\}});
 \item Definição (\comando{begin\{definicao\}[\ ]} ... \comando{begin\{definicao\}});
 \item demonstracao (\comando{begin\{demonstracao\}[\ ]} ... \comando{begin\{demonstracao\}}).
\end{itemize}

Abaixo temos um exemplo de proposição com sua demonstração:
\begin{proposicao}
 Sejam $a$ e $b$ reais, tais que $0<a<b$. Então $a^2<b^2$.
\end{proposicao}
\begin{demonstracao}
 Pela hipótese concluímos que $(b+a)>0$ e $(b-a)>0$.

Como $b^2-a^2=(b+a)(b-a)$ concluímos que $b^2-a^2>0$, ou seja, $a^2<b^2$.
\end{demonstracao}

Neste documento tratamos brevemente apenas dos ambientes mencionados anteriormente. Contudo, para escrever expressões matemáticas complexas é preciso estudar uma documentação mais específica como em \citeonline{cassagojr:1997:amslatex}.

Alguns dos ambientes matemáticos da classe \textit{icmc} podem ser usados também para outras finalidades como exemplos e definições.

\chapter{Listas}
\label{chapter:listas}
\section{Abreviaturas e Siglas}

A classe \textit{icmc} implementa a criação da lista de abreviaturas e siglas com o pacote \textit{nomencl}. A inserção de abreviaturas e siglas na lista é realizada com o comando \comando{sigla\{A\}\{B\}} que também insere o conteúdo da sigla no local do documento onde a mesma foi definida. Os parâmetros utilizados são: \textit{A} que é a sigla e \textit{B} que é o nome por extenso. Caso deseja-se inserir a sigla apenas na lista, pode-se utilizar o comando \comando{sigla*\{A\}\{B\}}.

Para se gerar a lista de siglas na parte pre-textual do documento é preciso incluir o comando \comando{incluidelistasiglas} antes do início do documento. Além disto, a compilação do documento deve conter o comando \textit{makeindex} após duas compilações com o \textit{pdflatex}. Por exemplo, supondo que o documento principal tenha o nome de \textit{thesis}, podemos usar a seguinte sequência de comandos:

\begin{verbatim}
pdflatex thesis.tex
pdflatex thesis.tex
makeindex thesis.nlo -s nomencl.ist -o thesis.nls
pdflatex thesis.tex
\end{verbatim}

No \autoref{chapter:ferramentas-uteis} serão apresentadas algumas ferramentas que podem facilitar o processo de compilação do documento. Em especial, o ShareLaTeX não necessita de um processo de compilação especial para gerar a lista de abreviaturas e siglas.


\section{Símbolos}

A definição de símbolos é semelhante a definição de siglas, porém deve ser usado o comando \comando{simbolo\{S\}\{DS\}}, onde \textit{S} é o símbolo e \textit{DS} é a descrição do símbolo. Como exemplo definimos os símbolos $\mathbb{X}$\simbolo{\mathbb{X}}{Variável X} e $\mathsf{I\!R}$\simbolo{\mathsf{I\!R}}{Conjunto dos números reais}. Para incluir a lista de símbolos, basta usar o comando \comando{incluidelistasimbolos} antes do início do documento.


\chapter{Ferramentas Úteis}
\label{chapter:ferramentas-uteis}
Existem diversas ferramentas para se trabalhar com \LaTeX. Três ferramentas que merecem destaque são o editor \textit{Texmaker} (\autoref{fig:texmaker}), o ShareLaTeX (\autoref{fig:sharelatex}) e o gerenciador de referências \textit{JabRef} (\autoref{fig:jabref}). Todas as ferramentas são livres e multiplataforma. 

\begin{figure}[htb]
\caption{Tela do Texmaker}
 \label{fig:texmaker}
 \centering
 \includegraphics[width=\textwidth]{texmaker.png}
 \fautor
\end{figure}

\begin{figure}[htb]
\caption{Site do ShareLaTeX}
 \label{fig:sharelatex}
 \centering
 \includegraphics[scale=0.5]{sharelatex.png}
 \fautor
\end{figure}

\begin{figure}[htb]
 \caption{Tela do JabRef}
 \label{fig:jabref}
 \centering
 \includegraphics[scale=0.45]{jabref.png}
\fautor
\end{figure}

O Texmaker pode ser obtido em \url{www.xm1math.net/texmaker} e o JabRef pode ser obtido em \url{jabref.sourceforge.ne}. É importante ressaltar que o Texmaker é apenas um editor, para compilar os documentos é necessário um ambiente \LaTeX instalado. Os ambientes Latex mais populares são o Texlive (\url{www.tug.org/texlive}) e o MiKTex (\url{miktex.org}). 

As estrutura de referências do bibtex utilizadas nesse \textit{template} contém alguns parâmetros adicionais que o modelo geral não tem, conforme pode ser consultado em \citeonline{abntex2cite-alf}. Desta forma, recomenda-se fortemente o uso do gerenciador de referências JabRef, uma vez que é possível customizá-lo para atender estas exigências. O código de customização pode ser visto no \autoref{chapter:configuracao-jabref}.

O ShareLaTeX é uma ferramenta de edição de documento em \LaTeX de forma online e está disponível em \url{www.sharelatex.com}. A ferramenta permite o compartilhamento e edição simultânea do conteúdo. Além disso, pode-se consultar o histórico da edições realizadas no documento. A principal vantagem de utilizar o ShareLaTeX é não precisar instalar o compilador para LaTeX.

\chapter{Citações e Referências}
\label{chapter:citacoes}
Em documentos acadêmicos podem existir citações diretas e citações indiretas. As citações indiretas são feitas quando se reescreve uma referência consultada. Nas citações indiretas há duas formatações possíveis dependendo de como ocorre a citação no texto. Quando o autor é mencionado explicitamente  deve ser usado o comando \comando{citeonline\{\}}, nas demais situações é usado o comando \comando{cite\{\}}. No quadro \ref{figura:citacao_indireta_explicita} encontrasse um  exemplo de uso do comando \comando{citeonline\{\}}.

\begin{quadro}[htb]
\caption{Exemplo de citação indireta explícita} \label{figura:citacao_indireta_explicita}
\hrulefill

\lstset{language=Tex, breaklines=true}
\begin{lstlisting}
Segundo \citeonline{silveira:2006}, o trabalho de conclusão de curso deve seguir as normas da ABNT.
\end{lstlisting}

\hrulefill

Segundo \citeonline{silveira:2006:manual_tcc}, o trabalho de conclusão de curso deve seguir as normas da ABNT.

\hrulefill

%\legend{Fonte: o autor.}
\end{quadro}

Para especificar a página consultada na referência é preciso acrescentá-la entre colchetes com os comandos \comando{cite[página]\{\}} ou \comando{citeonline[página]\{\}}. No quadro \ref{figura:citacao_indireta_pagina} é mostrado um exemplo de citação com página específica.

\begin{quadro}[htb]
\caption{Exemplo de citação indireta não explícita} \label{figura:citacao_indireta_pagina}
\hrulefill

\lstset{language=Tex, breaklines=true}
\begin{lstlisting}
A folha de aprovação é um elemento obrigatório na monografia de projeto final de curso trabalho de conclusão de curso.  \cite[p.~10]{silveira:2006}.
\end{lstlisting}

\hrulefill

A folha de aprovação é um elemento obrigatório no trabalho de conclusão de curso.  \cite[p.~10]{silveira:2006:manual_tcc}.

\hrulefill

\end{quadro}

As citações diretas acontecem quando o texto de uma referência é transcrito literalmente. As citações diretas são curtas (até três linhas) são inseridas no texto entre aspas duplas. Conforme exemplo no quadro \ref{figura:citacao_direta_curta}.

\begin{quadro}[htb]
\caption{Exemplo de citação direta curta}
\label{figura:citacao_direta_curta}
\hrulefill

\lstset{language=Tex, breaklines=true}
\begin{lstlisting}
``Os quadros, ao contrário das tabelas, apresentam dados textuais e devem localizar-se o mais próximo do texto a que se referem'' \cite[p.~25]{silveira:2006}.
\end{lstlisting}

\hrulefill

``Os quadros, ao contrário das tabelas, apresentam dados textuais e devem localizar-se o mais próximo do texto a que se referem'' \cite[p.~25]{silveira:2006:manual_tcc}.

\hrulefill
\end{quadro}

As citações longas (com mais de 3 linhas) podem ser inseridas via \comando{begin\{citacao\}} conforme quadro \ref{figura:citacao_direta_longa}.

\begin{quadro}[htb]
\caption{Exemplo de citação direta longa}
\label{figura:citacao_direta_longa}
\hrulefill

\lstset{language=Tex, breaklines=true}
\begin{lstlisting}
\begin{citacao}
Síntese final do trabalho, a conclusão constitui-se de uma resposta à hipótese enunciada na introdução. O autor manifestará seu ponto de vista sobre os resultados obtidos e sobre o alcance dos mesmos. Não se permite a inclusão de dados novos nesse capítulo nem citações ou interpretações de outros autores \cite[p.~25]{silveira:2006}.
\end{citacao}
\end{lstlisting}

\hrulefill

\begin{citacao}
Síntese final do trabalho, a conclusão constitui-se de uma resposta à hipótese enunciada na introdução. O autor manifestará seu ponto de vista sobre os resultados obtidos e sobre o alcance dos mesmos. Não se permite a inclusão de dados novos nesse capítulo nem citações ou interpretações de outros autores \cite[p.~25]{silveira:2006:manual_tcc}.
\end{citacao}

\hrulefill

\end{quadro}


% ---
% Finaliza a parte no bookmark do PDF, para que se inicie o bookmark na raiz
% ---
\bookmarksetup{startatroot}% 
% ---

% ----------------------------------------------------------
% ELEMENTOS PÓS-TEXTUAIS
% ----------------------------------------------------------
\postextual

% ----------------------------------------------------------
% Referências bibliográficas
% ----------------------------------------------------------
\bibliography{references}

% ---------------------------------------------------------------------
% GLOSSÁRIO
% ---------------------------------------------------------------------

% Arquivo que contém as definições que vão aparecer no glossário
\newword{WYSIWYG}{``What You See Is What You Get''  ou ``O que você vê é o que você obtém''.  Recurso tem por objetivo permitir que um documento, enquanto manipulado na tela, tenha a mesma aparência de sua utilização, usualmente sendo considerada final. Isso facilita para o desenvolvedor que pode trabalhar visualizando a aparência do documento sem precisar salvar em vários momentos e abrir em um \textit{software} separado de visualização}
\newword{Framework}{é uma abstração que une códigos comuns entre vários projetos de \textit{software} provendo uma funcionalidade genérica. \textit{Frameworks} são projetados com a intenção de facilitar o desenvolvimento de \textit{software}, habilitando designers e programadores a gastarem mais tempo determinando as exigências do \textit{software} do que com detalhes de baixo nível do sistema}

\newword{Template}{é um documento sem conteúdo, com apenas a apresentação visual (apenas cabeçalhos por exemplo) e instruções sobre onde e qual tipo de conteúdo deve entrar a cada parcela da apresentação}

\newword{Padrões de projeto}{ou \textit{Design Pattern}, descreve uma solução geral reutilizável para um problema recorrente no desenvolvimento de sistemas de \textit{software} orientados a objetos. Não é um código final, é uma descrição ou modelo de como resolver o problema do qual trata, que pode ser usada em muitas situações diferentes}

\newword{Web}{Sinônimo mais conhecido de \textit{World Wide Web} (WWW). É a interface gráfica da Internet que torna os serviços disponíveis totalmente transparentes para o usuário e ainda possibilita a manipulação multimídia da informação}

% Comando para incluir todas as definições do arquivo glossario.tex
\glsaddall
% Impressão do glossário
\printglossaries

% ----------------------------------------------------------
% Apêndices
% ----------------------------------------------------------

% ---
% Inicia os apêndices
% ---
\begin{apendicesenv}

    \chapter{Documento Básico Usando a Classe icmc}
    \label{chapter:documento-basico}
    
\definecolor{gray}{rgb}{0.4,0.4,0.4}
\definecolor{darkblue}{rgb}{0.0,0.0,0.6}
\definecolor{cyan}{rgb}{0.0,0.6,0.6}
\definecolor{maroon}{rgb}{0.5,0,0}
\definecolor{darkgreen}{rgb}{0,0.5,0}


\lstdefinelanguage{myLatex}
{
    keywords={\titulo},
    alsoletter={-},
    sensitive=false,
    morecomment=[l]{\%},
    morecomment=[s]{/*}{*/},
    morestring=[b]",
    morestring=[b]',
    keywordstyle=\bfseries\color{blue},
    commentstyle=\itshape\color{darkgreen},
    morekeywords={documentclass, titulo, autor, data, orientador, coorientador, curso, textoresumo, incluifichacatalografica, textodedicatoria*, textoagradecimentos*, textoepigrafe*, incluilistadefiguras, incluilistadetabelas, incluilistadequadros, incluilistadealgoritmos, incluilistadecodigos, incluilistadesiglas, incluilistadesimbolos, textual, chapter, postextual, begin, bibliography, end}, 
alsoletter={*, \{, \}, \[, \]},
 morekeywords=[2]{\{, \}, \[, \]},
 keywordstyle=[2]\bfseries\color{blue},
 moredelim=[s][\color{maroon}]{\{}{\}},
    moredelim=[s][\itshape\color{maroon}]{\[}{\]},
}

%\lstdefinelanguage{TeX}
%{
%moredelim=*[s][\color{maroon}]{\{}{\}}
%otherkeywords={\{, \}, \[, \], \\}
%  morestring=[b]",
%  moredelim=[s][\bfseries\color{maroon}]{<}{\ },
%  moredelim=[s][\bfseries\color{maroon}]{</}{>},
%  moredelim=[l][\bfseries\color{maroon}]{/>},
%  moredelim=[l][\bfseries\color{maroon}]{>},
%  commentstyle=\color{darkgreen},
%  stringstyle=\color{blue},
%  identifierstyle=\color{red},
%  keywordstyle=\bfseries\color{maroon}
%moredelim=[l][\bfseries\color{maroon}]{>},
%commentstyle=\color{darkgreen},
%  stringstyle=\color{blue},
%  identifierstyle=\color{red}, moredelim=[l][\bfseries\color{maroon}]{\{},
%  keywordstyle=\bfseries\color{maroon}
%}

%\lstset{language={[LaTeX]TeX},
%texcsstyle=*\bfseries\color{blue},
%keywordstyle=\bfseries\color{blue},
%commentstyle=\color{darkgreen},
%morecomment=[s][\color{red}]{\{}{\}},
%otherkeywords={$, \{, \}, \[, \]}
%}

%\begin{codigo}[caption={Exemplo de um documento básico}, label={codigo:documento-basico}, language={[LaTeX]TeX},  breaklines=true,morekeywords={titulo, autor, data, orientador, coorientador, curso, textoresumo, incluifichacatalografica, textodedicatoria*, textoagradecimentos*, textoepigrafe*, incluilistadefiguras, incluilistadetabelas, incluilistadequadros, incluilistadealgoritmos, incluilistadecodigos, incluilistadesiglas, incluilistadesimbolos, {\backslash}textual, chapter, postextual}, alsoletter={{\backslash},*},morecomment=[s][\color{red}]{\{}{\}}]
\begin{codigo}[caption={Exemplo de um documento básico}, label={codigo:documento-basico}, language={myLatex},  breaklines=true]
% Documento utilizando a classe icmc
% Opções: 
%   Qualificação          = qualificacao 
%   Curso                 = doutorado/mestrado
%   Situação do trabalho  = pre-defesa/pos-defesa (exceto para qualificação)
%   Versão para impressão = impressao
\ documentclass[doutorado, pos-defesa]{packages/icmc}

% Título do trabalho em Português
\tituloPT{Título da Monografia}

% Título do trabalho em Inglês
\tituloEN{Título da Monografia}

% Nome do autor
\autor[Abreviação]{Nome completo do autor}

% Gênero do autor (M ou F)
\genero{M}

% Data do depósito
\data{18}{12}{2012}

% Nome do Orientador
\orientador[Orientador]{Titulação do orientador}{Nome completo do Orientador}

% Nome do Coorientador (caso não exista basta remover)
\coorientador[Coorientador]{Titulação do coorientador}{Nome completo do Coorientador}
% Se coorientadora troque Coorientador: por Coorientadora dentro do colchetes

% Sigla do programa de Pós-graduação (CCMC, MAT, PIPGES, PROFMAT, MECAI)
\curso{CCMC}
% O valor entre colchetes é opcional para este programa

% Idioma principal do texto (EN ou PT)
\idioma{PT}

% Resumo
\textoresumo[Idioma]{
Texto do resumo do trabalho.
}{Lista de palavras-chave separada por virgulas}

% ----------------------------------------------------------
% ELEMENTOS PRÉ-TEXTUAIS
% ----------------------------------------------------------

% Inserir a ficha catalográfica
\incluifichacatalografica{tex/ficha-catalografica.pdf}

% Incluí o texto da Dedicatória
\textodedicatoria*{tex/pre-textual/dedicatoria}

% Incluí o texto dos Agradecimentos
\textoagradecimentos*{tex/pre-textual/agradecimentos}

% Incluí o texto da Epígrafe
\textoepigrafe*{tex/pre-textual/epigrafe}

% Inclui a lista de figuras
\incluilistadefiguras

% Inclui a lista de tabelas
\incluilistadetabelas

% Inclui a lista de quadros
\incluilistadequadros

% Inclui a lista de algoritmos
\incluilistadealgoritmos

% Inclui a lista de códigos
\incluilistadecodigos

% Inclui a lista de siglas e abreviaturas
\incluilistadesiglas

% Inclui a lista de símbolos
\incluilistadesimbolos

% Início do documento
\begin{document}

% ----------------------------------------------------------
% ELEMENTOS TEXTUAIS
% ----------------------------------------------------------
\textual

\chapter{Introdução}

Capítulo de Introdução

\chapter{Desenvolvimento}

Capítulo de Desenvolvimento

\chapter{Conclusão}

Capítulo de conclusão

% ----------------------------------------------------------
% ELEMENTOS PÓS-TEXTUAIS
% ----------------------------------------------------------
\postextual

% Nome do arquivo com as referências bibliográficas
\bibliography{referencias}

\end{document}

\end{codigo}

\end{apendicesenv}
% ---


% ----------------------------------------------------------
% Anexos
% ----------------------------------------------------------

% ---
% Inicia os anexos
% ---
\begin{anexosenv}

    \chapter{Páginas Interessantes na Internet} 
    \label{chapter:paginas-interessantes}
    \begin{description}
 \item[\url{http://www.tex-br.org}] Página em português com diversos tutoriais e referências interessantes sobre \LaTeX;
 \item[\url{http://en.wikibooks.org/wiki/LaTeX}] Livro em formato \textit{wiki} gratuito sobre \LaTeX;
 \item[\url{http://tobi.oetiker.ch/lshort/lshort.pdf}] Ótimo tutorial sobre \LaTeX (possui versão em português \url{http://alfarrabio.di.uminho.pt/~albie/lshort/ptlshort.pdf}, mas a versão em inglês é a mais atual);
 \item[\url{http://code.google.com/p/abntex2/}] Página do abnTeX2, grupo que desenvolve os pacotes e classes em \LaTeX para as normas da ABNT, nos quais a classe \textit{icmc} foi baseada;
\item[\url{http://www.more.ufsc.br}] Página do Mecanismo On-line para Referências  (MORE) desenvolvido pela UFSC;
\item[\url{http://detexify.kirelabs.org/classify.html}] Página para recuperar o código de símbolos em \LaTeX a partir do desenho fornecido pelo usuário.
 \end{description}

\end{anexosenv}
% ---

\end{document}